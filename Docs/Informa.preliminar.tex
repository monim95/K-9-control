\documentclass[12pt,letterpaper]{IEEEtran}
\usepackage[utf8]{inputenc}
\usepackage[spanish,es-tabla]{babel}
\usepackage{enumitem}
\usepackage{graphicx}
\usepackage{amssymb, amsmath, amsbsy}
\usepackage{upgreek}
\usepackage{mathrsfs}
\title{Proyecto Microcontroladores: K-9 Control}
\author{Mónica María Morales Jimenez / Dylan Ronan Sturm Trigueros}
\newcommand\MYhyperrefoptions{bookmarks=true,bookmarksnumbered=true,
pdfpagemode={UseOutlines},plainpages=false,pdfpagelabels=true,
colorlinks=true,linkcolor={black},citecolor={black},
urlcolor={black}}
\usepackage[\MYhyperrefoptions]{hyperref}


\begin{document}
\hypersetup{pdftitle={Proyecto Microcontroladores: K-9 Control},
pdfsubject={BINGE-61, Microcontroladores },
pdfauthor={Mónica María Morales Jimenez Dylan Ronan Sturm Trigueros},
pdfkeywords={Microcontroladores, K-9 control, microcontrolador}}
\renewcommand{\leftmark}{UNIVERSIDAD LATINA DE COSTA RICA -- BINGE-61 Microcontroladores}

\maketitle


\begin{abstract}
Informe previo del proyecto del curso.
\end{abstract}

\section{Descripción del proyecto}

Se realizará un sistema, en el cual, se podrá cuidar a una mascota, controlando la cantidad de alimento y de agua que es servido sin la necesidad de estar en la casa; todo esto, por medio del uso de Internet, permitiendo cumplir estas funciones esenciales. 

\section{Requerimientos}

\begin{itemize}
	\item Se debe controlar una bomba de agua, la cual, va a permitir un flujo de agua de un tanque hasta el recipiente de agua para la mascota.
	\item Control del servo motor, para permitir el paso del alimento que proviene del tanque de alimento hacia el recipiente.
	\item Construcción de una maqueta, que permita almacenar ambos tanques, tanto de alimento, como el de agua, que se pueda portar el microcontrolador que se va a utilizar y por último, los recipientes de alimento y agua.
	\item Obtener una medición del nivel actual de los tanques y recipientes, lo que nos permitira enviar alguna alerta sobre su estado actual; esto se debe realizar por medio de sensores.
	\item Conexion del dispositivo a Internet, por medio de un modulo que permita WiFi
	\item Aplicación donde permita el control de la cantidad de alimentos y agua que van a ser servidos y tener un control de la cantidad de agua y alimento que se encuentran en los recipientes y en los tanques. 
\end{itemize}
Opcionales:

\begin{itemize}
	\item En caso de ser necesario, la creacion de una pagina web para el uso del sistema.
	\item Usar una camara, para tener a disposicion la captura de imagenes o video de ser posible, de ser asi, se debe considerar un espacio para ubicarla en la maqueta.
\end{itemize}

\end{document}
